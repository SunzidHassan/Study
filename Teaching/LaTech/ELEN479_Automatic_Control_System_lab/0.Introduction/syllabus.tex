\documentclass[11pt, letter]{article}
\usepackage{hyperref}
\usepackage[margin = 1in]{geometry}
\usepackage{booktabs}
\usepackage{fontspec}
\setmainfont[Ligatures=TeX]{Times New Roman}
\usepackage{enumitem}

\setlength{\parindent}{0cm}
\setlength{\parskip}{0.25cm}

\begin{document}
	\scrollmode
	\pagenumbering{gobble}
	\begin{centering}
		\textbf{ELEN 479: Automatic Control Systems Laboratory}\\
		Section 001: NETH 218 – T 02:00 p.m. – 05:00 p.m.\\
		Section 002: NETH 218 – W 02:00 p.m. – 05:00 p.m.\\
		Fall 2023\\
	\end{centering}
	\vspace{0.5cm}
	
	\begin{tabular}{l r}
		\textbf{Instructor:}   & Dr. Hamidreza Mirzaei\\
		\textbf{Office:}       & Nethken Hall 217\\
		\textbf{Office Hours:} & TBD \\
		\textbf{Email:}		  & TBD\\
		\textbf{Phone:}	 	  & TBD\\
		\\
		\textbf{Grad. Assistant:} & Jordan Savoie\\
		\textbf{Office:}		     & TBD\\
		\textbf{Email:}		     & jss051@latech.edu\\
	\end{tabular}
	
	\textbf{Emergency Class Disruption Policy:} If a disaster or other emergency results in campus closure, this course will continue via Moodle and Zoom. You will be required to login to \href{http://moodle.latech.edu}{moodle.latech.edu} for further instructions. Please enroll in the Emergency Notification System (instructions below) to receive official campus updates. You may also refer to \href{http://ert.latech.edu}{ert.latech.edu} for updated information.

	\textbf{Course Objectives:} The purpose of this course is to apply the knowledge gained in ELEN 471 in carrying out experiments regarding the practical application of control theory with regard to electrical systems.
	
	\textbf{Textbook and Reading Material:} No textbook required. R.C. Dorf and R.H. Bishop, \textit{Modern Control Systems, 13th edition}, Pearson Prentice Hall, Upper Saddle River, NJ, 2016 may be utilized for supplemental information throughout the quarter. 
	
	\textbf{Required Software:} We will be using Matlab for the simulation labs. In addition to the base installation, you will need to install Simulink, MATLAB Support Package for Arduino Hardware, and the Control System Toolbox.
	
	\textbf{Disability Disclosure:} Students needing testing or classroom accommodations based on a disability are encouraged to discuss those needs with the instructors as soon as possible. For more information about eligibility for accommodations, contact the Department of Testing and Disability Services, 318-257-4221, \href{https://latech.edu/ods}{latech.edu/ods} for assistance.
	
	\textbf{Class Attendance:} This class will adhere to the guidelines for class attendance found detailed in Louisiana Tech University Policy 2206, available online at \href{https://latech.edu/administration/policies/p-2206/}{latech.edu/administration/policies/p-2206/}. Attendance will be taken promptly at the beginning of class and any student arriving after attendance has been checked without a reasonable excuse will be considered absent. Reasonable excuses DO NOT include routine doctor visits, car trouble, parking difficulties, oversleeping, or work schedules. Students will not receive any credit for missed labs and will lose 10 points on their final course grade for each unexcused absence.
	
	\textbf{Academic Honor Code:} In accordance with Louisiana Tech University’s Academic Honor Code, all students pledge the following: “Being a student of higher standards, I pledge to embody the principles of academic integrity.”

	\textbf{Academic Misconduct:} Academic behavior is governed by university policies and guidelines found in Louisiana Tech University’s Academic Honor Code, which can be found online at \href{https://catalog.latech.edu/content.php?catoid=17&navoid=733}{catalog.latech.edu/content.php\\?catoid=17\&navoid=733}. It is the student’s obligation to be familiar with and understand these policies, regulations, and guidelines.
	\begin{itemize}[nosep]
		\item Behavior: Students are expected to maintain a professional classroom environment. Students are to refrain from: verbal or physical violence, threats, improper language, disrespect to classmates and the instructor. Participants of such activities will be asked to leave the class. If you are removed from class for behavioral reasons, you will be considered absent for that class period. 
		\item Cheating: Cheating of any kind will not be tolerated. Collusion during examinations through verbal or electronic communication and/or plagiarism of any kind throughout the course will result in a zero grade for the assignment or exam as well as notification of the proper university officials. Multiple infractions will result in a zero grade for the course.
	\end{itemize}
	
	\textbf{Assignments:} Grading in this course will be based on reviewing the provided online resources and successfully implementing the required scripts and circuits, attending and conducting experiments during the regular lab meetings, and submitting formal laboratory reports for specific experiments. Details on each of these assignments will be provided on the course Moodle page. If there are any questions regarding the graded formal lab reports, the instructor should be contacted within one week after receiving the grade.
	
	\textbf{Grading:} The weighting of grades is as follows:
	\begin{itemize}[nosep]
		\item Correct script/circuit implementation (G) – 20 \%
		\item Laboratory notebook (I) – 20 \%
		\item Formal laboratory reports (G) – 60 \%
	\end{itemize}
	All assignments denoted with an (I) are to be completed individually. All assignments denoted with a (G) are to be completed collectively by your group.
	
	\textbf{Grading Scale:} The grading scale used for this course will be a traditional 10-point scale, as shown below:
	\begin{itemize}[nosep]
		\item A – above 90 \%
		\item B – 89.99-80 \%
		\item C – 79.99-70 \% 
		\item D – 69.99-60 \%
		\item F – below 60 \%
	\end{itemize}
	There will be no “curving” or rounding of grades offered in this course.
	
	\textbf{Syllabus Changes:} The contents of this syllabus are not expected to change. However, the instructor retains the right to interpretation and/or alteration of the policies contained herein. In the case of alteration, ample notice will be provided.
	
	\textbf{Emergency Notification System:} All Louisiana Tech students are strongly encouraged to enroll and update their contact information in the Emergency Notification System. It takes just a few seconds to ensure that you are able to receive important text and voice alerts in the event of a campus emergency. For more information on the Emergency Notification System, please visit \href{https://www.latech.edu/ens}{latech.edu/ens}
	
	\textbf{COVID-19 Policy:} Students who are feeling ill with COVID-19 symptoms, have been exposed to, or have tested positive for COVID-19 should not come to class and should report directly to the course instructor to make classroom absence arrangements. All absences related to COVID-19 will be handled in accordance with the attendance policy listed above.
	
	\textbf{Counseling Services:} Information and contact numbers and sites for Louisiana Tech Counseling Services are located at \href{https://www.latech.edu/counseling-services/}{latech.edu/counseling-services/}
	
	\textbf{AI Tools Usage Policy:} Artificial Intelligence (AI) applications, including but not limited to ChatGPT, offer transformative utilities that can be beneficial for engineering students in various aspects of this course. By enrolling in this course, you agree to adhere to the following guidelines detailed below concerning the ethical and responsible use of AI tools. 
	
	\emph{Permitted Uses}
	
	You are encouraged to employ AI tools in the following ways:
	\begin{itemize}[nosep]
		\item \textbf{Brainstorming and Idea Generation:} Use AI to generate ideas or as a creative catalyst for projects.
		\item \textbf{Course Topic Understanding:} Use AI tools to enhance learning about the fundamentals, theories, or practical aspects of course topics.
		\item \textbf{Problem-Solving Guidance:} Consult AI tools for step-by-step explanations to assist in solving analytical, programming, or circuit design problems.
		\item \textbf{Writing and Grammar Assistance:} Use AI tools to improve syntax, structure, and quality of written assignments, provided the final work is substantially your own.
	\end{itemize}
	
	\emph{Prohibited Uses}
	
	AI tools may not be used in the following scenarios:
	\begin{itemize}[nosep]
		\item \textbf{Impersonation:} To impersonate you in any academic or classroom contexts, including online forums or correspondence.
		\item \textbf{Plagiarism:} To generate entire sentences, paragraphs, or larger bodies of text that you submit as your own original work.
		\item \textbf{Assignment Drafting:} To automatically create drafts for writing, presentation, or discussion assignments.
		\item \textbf{Direct Homework Assistance:} You may consult AI to understand how to approach an assignment, but you must complete the actual work independently.
		\item \textbf{Exams and Quizzes:} Using AI tools to assist you during closed-book exams and quizzes is strictly prohibited.
	\end{itemize}
	
	\emph{Citing AI Usage}
	
	If you utilize AI tools for your assignments or projects, you are required to explicitly reference and describe how the AI was involved. These references should be included in the "References" or "Acknowledgments" section at the end of your assignments. Here is a suggested citation format:
	
	\textit{Information or assistance obtained via Artificial Intelligence tools including OpenAI's ChatGPT. (2023).}

 	
	\centering
	\textbf{Tentative Course Schedule:}
	
	\begin{tabular}{c c c l}
		\toprule
		Class & 001      & 002      & Topic\\
		\midrule
		1     & 09/12/23 & 09/13/23 & Course Introduction \& Orientation\\
		2     & 09/19/23 & 09/20/23 & Modeling Control Systems in Matlab/Simulink\\
		3     & 09/26/23 & 09/27/23 & Controlling an RLC Circuit\\
		4     & 10/03/23 & 10/04/23 & Modeling State Space Systems\\
		5     & 10/10/23 & 10/11/23 & System ID\\
		6     & 10/17/23 & 10/18/23 & Controller Design Using Root Locus\\
		7     & 10/24/23 & 10/25/23 & Introduction to Motor Control\\
		8     & 10/31/23 & 11/01/23 & Motor Control with Root Locus\\
		9     & 11/07/23 & 11/08/23 & Motor Control with Analog PID\\
		\bottomrule
	\end{tabular}


	\batchmode
\end{document}