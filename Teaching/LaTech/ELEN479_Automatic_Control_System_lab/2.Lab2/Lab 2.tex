\documentclass[12pt]{article}

\usepackage{hyperref}
\usepackage[xetex, dvipsnames]{xcolor}
\usepackage{fontspec}
\setmainfont{Times New Roman}
\usepackage[margin = 2cm]{geometry}
\usepackage{circuitikz}
\usepackage{siunitx}
\usepackage{graphicx}
\usepackage{enumitem}
\usepackage{placeins}

\setlength{\parindent}{0cm}
\setlength{\parskip}{\baselineskip}

\newcommand{\header}[1]{\textcolor{CornflowerBlue}{#1}}
\renewcommand{\title}[1]{\header{\large #1}}

\begin{document}
	\scrollmode
	\begin{centering}
		\textbf{Louisiana Tech University\\
			ELEN 479 Automatic Control Systems Lab\\
			Lab 2\\}
	\end{centering}
	\vspace{\baselineskip}
	\title{CONTROLLING A REAL RLC CIRCUIT}
	
	\header{Objective:}
	
	You already simulated the step response of an RLC circuit with a negative feedback controller last week. This week, you will implement that system using physical components.
	
	\FloatBarrier
	\header{Theory:}
	
	Review lab 1 for more information on relevant control theory and how to derive the the governing equation for this system. Figure~\ref{fig:RLC circuit} shows our system and and Equation~\ref{eq:RLC circuit} is the governing equation for it.
	\begin{figure}[hbt]
		\centering
		\begin{circuitikz}[american]
			\draw (0,0) to[R, label = R] (4,0) to[short, -o] (7,0) node[below]{$+$};
			\draw (4,0) to[C, label = C] (4,-3);
			\draw (0,-3) to[L, label = L] (4,-3) to[short, -o] (7,-3) node[above]{$-$};
			\draw (0,0) to[vsource, label = $V_\mathrm{in}(s)$] (0,-3);
			\draw (7,-1.5) node[]{$V_\mathrm{out}(s)$};
		\end{circuitikz}
		\caption{\label{fig:RLC circuit} An RLC circuit}
	\end{figure}
	\begin{equation} \label{eq:RLC circuit}
		\frac{V_\mathrm{out}(s)}{V_\mathrm{in}(s)} = \frac{1}{C\left(Ls^2 + Rs +\frac{1}{C}\right)}
	\end{equation}
	
	There are many ways to implement a feedback loop for this system using op-amps. We will use a differencing amplifier to perform the actual feedback and a non-inverting buffer to read the output.
	
	\begin{figure}[hbt]
		\centering
		\begin{circuitikz}
			\draw (0,0) node[left]{$v_2$} to[R,o-,label=$R$] (3,0) -- (3.5,0) node[op amp, anchor = -](OA){};
			\draw (OA.+) -- ++(-0.5,0) to[R, -o, label=$R$] ++(-3,0) node[left]{$v_1$};
			\draw (OA.+) ++(-0.5,0) to[R, label=$R$] ++(0,-2) node[ground]{};
			\draw (3,0) -- ++(0,1.5) coordinate(c) to[R, label=$R$] (c -| OA.out) -- (OA.out) to[short, -o] ++(1,0) node[right]{$v_\mathrm{out}$};
		\end{circuitikz}
		\caption{\label{fig:differencing op-amp} A differencing op-amp configuration}
	\end{figure}
	
	The governing equation for the circuit in Figure~\ref{fig:differencing op-amp} is given by Equation~\ref{eq:differencing op-amp} below. All resistors are approximately the same value, $R$.
	\begin{equation}\label{eq:differencing op-amp}
		v_1 - v_2 = v_\mathrm{out}
	\end{equation}
	
	\begin{figure}[hbt]
		\centering
		\begin{circuitikz}
			\draw (0,0) node[op amp, anchor = -](OA){};
			\draw (OA.+) to[short,-o] ++(-1,0) node[left]{$v_\mathrm{in}$};
			\draw (0,0) -- ++(-0.5,0) -- ++(0,1.5) coordinate(c) -- (c -| OA.out) -- (OA.out) to[short, -o] ++(1,0) node[right]{$v_\mathrm{out}$};
		\end{circuitikz}
		\caption{\label{fig:buffering op-amp} A buffering op-amp configuration}
	\end{figure}
	
	The circuit in Figure~\ref{fig:buffering op-amp} is governed by Equation~\ref{eq:buffering op-amp}. It's important that we use a buffer to read the output when we do not want to load a circuit significantly. It's best to use a non-inverting op-amp configuration as a buffer because the input signal goes into the high-impedance port of the op-amp, which means virtually no current will enter the op-amp.
	\begin{equation}\label{eq:buffering op-amp}
		v_\mathrm{out} = v_\mathrm{in}
	\end{equation}
	\newpage
	\header{Lab Instructions}
	
	\header{Task 1:}
	\FloatBarrier
	
	\begin{figure}[hbt]
		\centering
		\begin{circuitikz}
			\draw (1.5, 0) node[adder](ADD){};
			\draw (0,0) node[left]{Control Voltage} to[short, o-] (ADD.west) (ADD.east) to[twoport, l = RLC] (6,0) to[short, *-] (6,-2) coordinate(c) -- (c -| ADD.south) -- (ADD.south);
			\draw (6,0) to[short, -o] (7,0) node[right]{Output};
			\draw (ADD.west) node[above left]{$+$};
			\draw (ADD.south) node[below right]{$-$};
		\end{circuitikz}
		\caption{System block diagram}
	\end{figure}

	Using a \href{https://www.ti.com/lit/ds/symlink/lf353-n.pdf}{LM353 dual op-amp} (Fig.~\ref{fig:lm353}) in a differencing configuration, add negative feedback to an RLC circuit made using a 100~\si{\micro\henry} inductor, a 1~\si{\micro\farad} capacitor, and a 510~\si{\ohm} resistor. You will need additional 10~\si{\kilo\ohm} resistors for the op-amp. Make sure to draw a circuit diagram of your overall system.
	
	Remember to power your op-amp using $\pm$5~V to the positive and negative power pins. Although we show the capacitor in Figure~\ref{fig:RLC circuit} to be the second component of the loop, the order does not affect the governing equation. This means you can order the components in whichever way makes implementing the system most convenient.
	
	\begin{figure}[hbt]
		\centering
		\includegraphics[width = 0.5\textwidth]{lm353.png}
		\caption{\label{fig:lm353} LM353 op-amp pin-out}
	\end{figure}
	
	\newpage
	\header{Task 2:}
	
	Connect the arbitrary waveform generator to the input and read the input and the output with your oscilloscope. Run a 1~Vpp squarewave with a frequency of 75~\si{\hertz} into the system.
	
	\begin{itemize}[nosep]
		\item Rerun your Matlab script from last week, but with the values of R, L, and C changed to match your current components. How do the simulation and the real results differ? You may need to play with the time and voltage scales to get a similar view to your simulation.
		\item But wait! The differrencing op-amp has a input impedance of 10~\si{\kilo\ohm}. That might affect the circuit. Try buffering the output as it goes into the negative terminal of the differencing amplifier using the second channel of your op-amp chip. Does the system output better match the simulation? Was there any significant change in the waveform?
	\end{itemize}
	
		\begin{figure}[hbt]
		\centering
		\begin{circuitikz}
			\draw (1.5, 0) node[adder](ADD){};
			\draw (0,0) node[left]{Control Voltage} to[short, o-] (ADD.west) (ADD.east) to[twoport, l = RLC] (6,0) to[short, *-] (6,-2) coordinate(c) to[amp, l = {$a=1$}] (c -| ADD.south) -- (ADD.south);
			\draw (6,0) to[short, -o] (7,0) node[right]{Output};
			
			\draw (ADD.west) node[above left]{$+$};
			\draw (ADD.south) node[below right]{$-$};
		\end{circuitikz}
		\caption{System block diagram with buffer}
	\end{figure}
	
	\header{Task 3:}
	
	Try applying different signals to the input, such as sine, ramp, or something from the arbitrary waveform menu.
	
	\begin{itemize}[nosep]
		\item How well does the output follow the input?
		\item Try using a larger resistor for R in your circuit. What happens?
		\item How does using a smaller resistor affect this?
	\end{itemize}
	
	\header{Post Lab:}
	\begin{enumerate}[nosep]
		\item Why does changing the size of the resistor change how quickly the controller can react to changes?
		\item Play around in Matlab, how much do you have to change the values of the RLC circuit to get significant overshoot and settling time?
		\item How might you construct a sensor with op-amps to measure any voltage at any point of a system without affecting load?
	\end{enumerate}
	\batchmode
\end{document}	
